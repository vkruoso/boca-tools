\documentclass[12pt,oneside]{article} % Uma Coluna e lingua portuguesa
%\usepackage[T1]{fontenc}        % Permite digitar os acentos de forma normal
\usepackage[utf8]{inputenc}
%\usepackage[english]{babel}
\usepackage[portuges,brazil]{babel}
%\usepackage[latin1]{inputenc}
\usepackage[dvips]{graphicx}    % Permite Gráficos
%\usepackage{times}    % Fonte Times
\usepackage{fancyhdr}
\usepackage{array}
\usepackage{multicol}
\usepackage[colorlinks=true,linkcolor=blue,urlcolor=blue]{hyperref}
\usepackage{nomencl}    % glossario
\usepackage{amssymb}
\usepackage{amsmath}
\usepackage[compact]{titlesec}

%=======================================================================

% Hifenização das palavras desconhecidas pelo LaTeX
%\hyphenation{}
\paperheight    297mm
\paperwidth     210mm
\voffset         -15mm
\headheight      15pt %% tamanho de letra
\headsep         5mm  %% para o início do texto
\oddsidemargin  -3.0mm
\evensidemargin -3.0mm
\textwidth      167.0mm
\topmargin      005.0mm
\textheight     240.0mm
\footskip       10.0mm



\title{TITLE}

\author{Maratona de Programação}
\title{TITLE}
\date{DATE}
\usepackage{indentfirst}
\usepackage{subfig}

\parindent=0pt
\setlength{\parskip}{7pt plus 1pt minus 2pt}
\titlespacing{\section}{0pt}{*0}{*0}
\titlespacing{\subsection}{0pt}{*0}{*0}
\titlespacing{\subsubsection}{0pt}{*0}{*0}

\begin{document}

\begin{center}
\textbf{\Huge TITLE} \\
\vspace{0.2cm}
\textit{DATE} \\
\vspace{1.0cm}
\textbf{Sevidor BOCA:} \\
\texttt{\large BOCAURL} \\
\vspace{1.0cm}
\begin{figure}[h!]
	\centering
 \includegraphics[scale=0.5]{capa.png}
\end{figure}
\vspace{1.0cm}
\textbf{Organizadores:}\\
{\small STAFF} \\
\vspace{1.0cm}
\vspace{1.0cm}
\textbf{Lembretes:} \\
\end{center}
{\scriptsize
\begin{itemize}
 \item É permitido consultar livros, anotações ou qualquer outro material
        impresso durante a prova.

 \item A correção é automatizada, portanto, siga atentamente as exigências
        da tarefa quanto ao formato da entrada e saída de seu programa.
        Deve-se considerar entradas e saídas padrão.

 \item Procure resolver o problema de maneira eficiente. Se o tempo superar
        o limite pré-definido, a solução não é aceita.  As soluções são
        testadas com outras entradas além das apresentadas como exemplo dos
        problemas.

 \item Teste seu programa antes de submetê-lo. A cada problema detectado
        (erro de compilação, erro em tempo de execução, solução incorreta,
        formatação imprecisa, tempo excedido $\dots$), há penalização de
        $20$ minutos. O tempo é critério de desempate entre duas ou mais
        equipes com a mesma quantidade de problemas resolvidos.

 \item Utilize o \emph{clarification} para dúvidas da prova. Os juízes podem
        opcionalmente atendê-lo com respostas acessíveis a todos.
\end{itemize}
}

\clearpage

\pagestyle{fancy}
\renewcommand{\footrulewidth}{0.7pt}
\renewcommand{\headrulewidth}{0.7pt}
\lhead{TITLE}
%\chead{Maratonas de Programação}
\rhead{DATE}
\cfoot{\thepage}



\newpage

\newpage

PROBLEMS

\end{document}

Alice has recently learned to use the Sieve of Eratosthenes, an ancient
algorithm for finding all prime numbers up to any given limit. As expected, she
was really impressed by it's simplicity and elegancy.

Now, she has decided to design her own sieve method: The Sieve of Alice,
    formally defined by the following procedure, which determines the Sieve of
    Alice up to a given limit $N$.

\begin{enumerate}
\item Create a list of consecutive integers from $N$ to $2$ ($N$, $N-1$, $N-2$,
    ..., $3$, $2$). All of those $N-1$ numbers are initially unmarked.

\item Initially, let $P$ equal $N$, and leave this number unmarked.

\item Mark all the proper divisors of $P$ (i.e. $P$ remains unmarked).

\item Find the largest unmarked number from $2$ to $P - 1$, and now let $P$ equal this number.

\item If there were no more unmarked numbers in the list, stop. Otherwise,
    repeat from step $3$.  Unfortunately, Alice has not found an useful
    application for it's Sieve. But she still wants to know, for a given limit
    $N$, how many integers will remain unmarked.
\end{enumerate}

\subsection*{Input}

The first line contains an integer $T$, the number of test cases ($1 \leq T
    \leq 10^4$). Each of the next $T$ lines contains an integer $N$ ($2 \leq N
    \leq 10^6$).

\subsection*{Output}

Output $T$ lines, one for each test case, containing the required answer.

\begin{table}[!h]
\centering
\begin{tabular}{|l|l|}
\hline
\begin{minipage}[t]{3in}
\textbf{Sample Input}
\begin{verbatim}
3
2
3
5
\end{verbatim}
\vspace{1mm}
\end{minipage}
&

\begin{minipage}[t]{3in}
\textbf{Sample Output}
\begin{verbatim}
1
2
3
\end{verbatim}
\vspace{1mm}
\end{minipage} \\
\hline
\end{tabular}
\end{table}

\newpage

Uma string $x$ é um anagrama da string $y$ se podemos reordenar as letras da string
$x$ e formar exatamente a string $y$. Por exemplo, a string ``DOG'' é um anagrama de
``GOD'', assim como ``BABA'' é anagrama de ``AABB''. A string ``ABBAC'' não é um
anagrama de ``CAABA''.

São dadas duas strings $s$ e $t$, que têm o mesmo tamanho e contém apenas letras
maiúsculas do alfabeto inglês (A,B,C, ..., X, Y, Z). Através da aplicação
sucessiva
de operações, você deve transformar a string $s$ em algum anagrama da string $t$.

Uma operação sobre uma string consiste em trocar qualquer (mas apenas uma) letra da
string por qualquer outra letra do alfabeto inglês (A, ..., Z).

Transforme a string $s$ em um anagrama da string $t$ com o menor número possível de
operações. Se você consegue obter vários anagramas de $t$ com o mínimo de
operações, obtenha o lexicograficamente menor. A ordem lexicográfica de strings
é a conhecida ordem ``de dicionário'', ou ``alfabética''.

\subsection*{Input}

Cada caso de teste consiste em duas linhas. A primeira linha contém a string $s$,
e a segunda contém a string $t$. As strings têm o mesmo tamanho (de
$1$ a $10^5$ caracteres) e consiste apenas de letras maiúsculas (A-Z).
O último caso de teste é seguido por duas linhas contendo $\#$.

\subsection*{Output}

Para cada caso de teste, imprima duas linhas. Na primeira, imprima $z$ - o número
mínino de operações a serem aplicadas sobre $s$ para se obter um anagrama de $t$. Na
segunda linha, imprima o anagrama obtido com $z$ operações. Lembre-se que o
anamagra obtivo deve ser o menor possível, lexicograficamente.

\begin{table}[!h]
\centering
\begin{tabular}{|l|l|}
\hline
\begin{minipage}[t]{3in}
\textbf{Exemplo de entrada}
\begin{verbatim}
ABA
CBA
CDBABC
ADCABD
AABAA
BBAAA
#
#
\end{verbatim}
\vspace{1mm}
\end{minipage}
&

\begin{minipage}[t]{3in}
\textbf{Exemplo de saída}
\begin{verbatim}
1
ABC
2
ADBADC
1
AABAB
\end{verbatim}
\vspace{1mm}
\end{minipage} \\
\hline
\end{tabular}
\end{table}

\newpage

When John was a little kid he didn't have much to do. There was no internet, no
Facebook, and no programs to hack on. So he did the only thing he could... he
evaluated the beauty of strings in a quest to discover the most beautiful
string in the world.

Given a string $s$, little Johnny defined the beauty of the string as the sum
of the beauty of the letters in it.

The beauty of each letter is an integer between $1$ and $26$, inclusive, and no two
letters have the same beauty. Johnny doesn't care about whether letters are
uppercase or lowercase, so that doesn't affect the beauty of a letter.
(Uppercase '$F$' is exactly as beautiful as lowercase '$f$', for example.)

You're a student writing a report on the youth of this famous hacker. You found
the string that Johnny considered most beautiful. What is the maximum possible
beauty of this string?

\subsection*{Input}

The input file consists of a single integer $m$ followed by $m$ lines.

\subsection*{Output}

Your output should consist of, for each test case, a line containing the string
"Case \#$x$: $y$" where $x$ is the case number (with 1 being the first case in
    the input file, 2 being the second, etc.) and $y$ is the maximum beauty for
that test case.

\subsection*{Constraints}
$5 \leq m \leq 50$

$2 \leq$ length of $s \leq 500$

\begin{table}[!h]
\centering
\begin{tabular}{|l|l|}
\hline
\begin{minipage}[t]{4.2in}
\textbf{Sample Input}
\begin{verbatim}
5
ABbCcc
Good luck in the Facebook Hacker Cup this year!
Ignore punctuation, please :)
Sometimes test cases are hard to make up.
So I just go consult Professor Dalves
\end{verbatim}
\vspace{1mm}
\end{minipage}
&

\begin{minipage}[t]{1.8in}
\textbf{Sample Output}
\begin{verbatim}
Case #1: 152
Case #2: 754
Case #3: 491
Case #4: 729
Case #5: 646
\end{verbatim}
\vspace{1mm}
\end{minipage} \\
\hline
\end{tabular}
\end{table}

\newpage

Cuca saiu para jogar futebol com os amigos e esqueceu do encontro que tinha com
a namorada. Ciente da mancada, Cuca deseja elaborar um pedido especial de
desculpas. Resolveu então enviar flores e usar o cartão da floricultura para
escrever um pedido especial de desculpas.

Cuca buscou na internet um conjunto de frases bonitas contendo a palavra
``desculpe'' (que pode ocorrer mais de uma vez na mesma frase). No entanto, o
cartão da floricultura é pequeno, e nem todas as frases que Cuca colecionou
poderão ser aproveitadas.

Cuca quer aproveitar o espaço do cartão, onde cabe um número limitado de
caracteres, para escrever um sub-conjunto das frases coletadas de modo que
apareça o máximo de vezes possível a palavra ``desculpe''.

Escreva um programa que, dados o número de caracteres que cabem no cartão e a
quantidade de frases coletadas (com os respectivos comprimentos e os números de
ocorrências da palavra ``desculpe''), determine o número máximo de vezes
que a palavra aparece, utilizando apenas as frases colecionadas, sem
repetí-las.


\subsection*{Entrada}

A entrada é constituída de vários casos de teste. A primeira linha de um caso de
teste contém dois números inteiros $C$ e $F$ indicando respectivamente o comprimento
do cartão em caracteres $(8 \leq C \leq 1000)$ e o número de frases coletadas
$(1 \leq F \leq 50)$. Cada uma das $F$ linhas seguintes descreve uma frase coletada. A
descrição é composta por dois inteiros $N$ e $D$ que indicam respectivamente o
número de caracteres na frase $(8 \leq N \leq 200)$ e quantas vezes a palavra
``desculpe'' ocorre na frase $(1 \leq D \leq 25)$.

O final da entrada é indicado por $C = F = 0$.


\subsection*{Saída}

Para cada caso de teste seu programa deve produzir três linhas na saída. A
primeira identifica o conjunto de teste no formato \textit{Teste n}, onde $n$ é numerado
a partir de $1$. A segunda linha deve conter o máximo número de vezes que a
palavra ``desculpe'' pode aparecer no cartão, considerando que apenas frases
coletadas podem ser utilizadas, e cada frase não é utilizada mais de uma vez. A
terceira linha deve ser deixada em branco. A grafia mostrada no Exemplo de
Saída, abaixo, deve ser seguida rigorosamente.

\begin{table}[!ht]
\centering
\begin{tabular}{|l|l|}
\hline
\begin{minipage}[t]{3in}
\textbf{Exemplo de entrada}
\begin{verbatim}
200 4
100 4
100 1
120 2
80 5
40 3
10 1
10 1
20 2
0 0
\end{verbatim}
\vspace{1mm}
\end{minipage}
&

\begin{minipage}[t]{3in}
\textbf{Exemplo de saída}
\begin{verbatim}
Teste 1
9

Teste 2
4

\end{verbatim}
\vspace{1mm}
\end{minipage} \\
\hline
\end{tabular}
\end{table}

\newpage

\vspace*{\fill}

\newpage


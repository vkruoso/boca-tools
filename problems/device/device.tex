Manao é responsável por gerenciar os equipamentos computacionais de sua
universidade. Um certo dia, Manao recebeu a notícia de que a universidade comprou um
novo equipamento, e de que ele ficará responsável por sua manutenção.

O equipamento comprado consiste em um dispositivo que recebe como entrada duas
placas contendo sequências de \textit{bits} de tamanho $M$ cada e imprime como saída uma outra
placa contendo uma sequência de tamanho $M$. O $i-$ésimo \textit{bit} da sequência de saída é igual a
um \textbf{e} (\textit{AND}), um \textbf{ou} (\textit{OR}) ou um
\textbf{ou-exclusivo} (\textit{XOR}) dos $i-$ésimos \textit{bits} das sequências
de entrada. A operação realizada em cada \textit{bit} pode ser distinta.

Por exemplo, suponha que um dispositivo recebe como entrada placas com sequências de tamanho $M = 4$
e que o dispositivo corresponde, nesta ordem, às operações \textit{(AND, OR, AND, XOR)}. Se
alimentado com placas com as sequências $(1,0,1,1)$ e $(0,1,1,1)$, o dispositivo
produzirá como saída uma placa com a sequência $(0,1,1,0)$.

Manao sabe o valor de $M$, mas não sabe a configuração do dispositivo comprado, isto é,
quais operações são realizadas em cada posição das entradas. Manao tem a sua
disposição $N$ placas contendo sequências de $M$ \textit{bits} cada. Quando o
dispositivo chegar, Manao quer descobrir exatamente qual operação o dispositivo
realiza em cada posição.

Para tal, Manao pode escolher qualquer par de placas distintas das $N$ placas
disponíveis, usá-las como entrada para o
dispositivo e analisar sua saída. Manao pode fazer isto quantas vezes forem
necessárias, com quantos pares de placas forem necessários.
Sua tarefa é ajudar Manao e informar se o conjunto de $N$ placas que Manao
possui é suficiente para determinar exatamente a configuração do dispositivo,
independente da configuração do dispositivo que chegar.

\subsection*{Entrada}

A entrada consiste em um ou mais casos de teste. Cada caso começa com uma linha
contendo $N$ $(1 \leq N \leq 50)$ e $M$ $(1 \leq M \leq 50)$. As próximas $N$
linhas descrevem as placas que Manao tem. Cada placa é descrita por uma linha
contendo $M$ dígitos. É garantido que todos os digitos são iguais a $0$ ou $1$.

O último caso de teste é seguido por uma linha contendo dois zeros.

\subsection*{Saída}

Para cada caso de teste, imprima \textit{YES} se as placas de Manao são
suficientes para determinar quais são as operações realizadas pelo dispositivo, e imprima \textit{NO} caso contrário.

\newpage

\begin{table}[!h]
\centering
\begin{tabular}{|l|l|}
\hline
\begin{minipage}[t]{3in}
\textbf{Exemplo de Entrada}
\begin{verbatim}
3 3
010
011
000
4 1
1
0
1
0
1 5
11111
4 7
0110011
0101001
1111010
1010010
5 9
101001011
011011010
010110010
111010100
111111111
0 0
\end{verbatim}
\vspace{1mm}
\end{minipage}
&

\begin{minipage}[t]{3in}
\textbf{Exemplo de Saída}
\begin{verbatim}
NO
YES
NO
NO
YES
\end{verbatim}
\vspace{1mm}
\end{minipage} \\
\hline
\end{tabular}
\end{table}

\newpage

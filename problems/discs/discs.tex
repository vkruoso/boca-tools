Operating systems are large software artefacts composed of many packages, usually
distributed on several media, e.g., discs. You probably remember the time when your
favourite operating system was delivered on 21 floppy discs, or, a few years later,
on 6 CDs. Nowadays, it will be shipped on several DVDs, each containing tens of
thousands of packages.

The installation of certain packages may require that other packages have been
installed previously. Therefore, if the packages are distributed on the media in
an unsuitable way, the installation of the complete system requires you to perform
many media changes, provided that there is only one reading device available,
e.g., one DVD-ROM drive. Since you have to start the installation somehow, there
will of course be one or more packages that can be installed independently of all
other packages.

Given a distribution of packages on media and a list of dependences between
packages, you have to calculate the minimal number of media changes required
to install all packages. For your convenience, you may assume that the operating
system comes on exactly 2 DVDs.

\subsection*{Input}

The input contains several test cases. Every test case starts with three
integers N1, N2, D. You may assume that $1 \leq N_1 , N_2 \leq 50000$ and
$0 \leq D \leq 100000$.
The first DVD contains $N_1$ packages, identified by the numbers $1, 2, ..., N_1$.
The second DVD contains $N_2$ packages, identified by the numbers
$N_1+1, N_1+2, ..., N_1+N_2$. Then follow D dependence specifications, each
consisting of two integers $x_i$, $y_i$. You may assume that $1 \leq x_i,y_i \leq N_1+N_2$
for $1 \leq i \leq D$. The dependence specification means that the installation of
package $x_i$ requires the previous installation of package $y_i$. You may assume
that there are no circular dependences. The last test case is followed
by three zeros.

\subsection*{Output}

For each test case output on a line the minimal number of DVD changes
required to install all packages. By convention, the DVD drive is empty
before the installation and the initial insertion of a disc counts as one
change. Likewise, the final removal of a disc counts as one change,
leaving the DVD drive empty after the installation.

\begin{table}[!h]
\centering
\begin{tabular}{|l|l|}
\hline
\begin{minipage}[t]{3in}
\textbf{Sample Input}
\begin{verbatim}
3 2 1
1 2
2 2 2
1 3
4 2
2 1 1
1 3
0 0 0
\end{verbatim}
\vspace{1mm}
\end{minipage}
&

\begin{minipage}[t]{3in}
\textbf{Sample Output}
\begin{verbatim}
3
4
3
\end{verbatim}
\vspace{1mm}
\end{minipage} \\
\hline
\end{tabular}
\end{table}

\newpage

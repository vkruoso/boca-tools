The SUM problem can be formulated as follows: given four lists A, B, C, D of
integer values, compute how many quadruplet $(a,b,c,d) \in A \times B \times C \times D$
are such that $a+b+c+d=0$. In the following, we assume that all lists have
the same size n.

\subsection*{Input}

The input contains several test cases. For each test case the first line
of the input contains the size of the lists $N$ (this value
can be as large as 4000). We then have $N$ lines containing four integer values
(with absolute value as large as $2^{28}$) that belong respectively to A, B, C and D.


\subsection*{Output}

For each test case your program has to write on line with the number
of quadruplets whose sum is zero. The last test case is followed by a
line containing one zero.

\begin{table}[!h]
\centering
\begin{tabular}{|l|l|}
\hline
\begin{minipage}[t]{3in}
\textbf{Exemplo de entrada}
\begin{verbatim}
6
-45 22 42 -16
-41 -27 56 30
-36 53 -37 77
-36 30 -75 -46
26 -38 -10 62
-32 -54 -6 45
0
\end{verbatim}
\vspace{1mm}
\end{minipage}
&

\begin{minipage}[t]{3in}
\textbf{Exemplo de saída}
\begin{verbatim}
5
\end{verbatim}
\vspace{1mm}
\end{minipage} \\
\hline
\end{tabular}
\end{table}

\newpage

HQ9+ is a joke programming language which has only four one-character instructions:

\begin{itemize}
    \item "H" prints "Hello, World!",

    \item "Q" prints the source code of the program itself,

    \item "9" prints the lyrics of "99 Bottles of Beer" song,

    \item "+" increments the value stored in the internal accumulator.
\end{itemize}

Instructions "H" and "Q" are case-sensitive and must be uppercase. The
characters of the program which are not instructions are ignored.

You are given a program written in HQ9+. You have to figure out whether
executing this program will produce any output.

\subsection*{Input}

The first line of the input contains the integer $T$ ($1 \leq T \leq 100$),
the number of test cases.
Each of the following T lines contains a test case, a program in HQ9+.
Each line will contain between 1 and 100 characters, inclusive. ASCII-code of
each character will be between 33 (exclamation mark) and 126 (tilde),
     inclusive.

\subsection*{Output}

For each test case, output "YES", if executing the program will produce any
output, and "NO" otherwise.

\begin{table}[!h]
\centering
\begin{tabular}{|l|l|}
\hline
\begin{minipage}[t]{3in}
\textbf{Sample Input}
\begin{verbatim}
2
Hi!
Codeforces
\end{verbatim}
\vspace{1mm}
\end{minipage}
&

\begin{minipage}[t]{3in}
\textbf{Sample Output}
\begin{verbatim}
YES
NO
\end{verbatim}
\vspace{1mm}
\end{minipage} \\
\hline
\end{tabular}
\end{table}

\subsection*{Note}
In the first case the program contains only one instruction - "H", which prints "Hello, World!".

In the second case none of the program characters are language instructions.

\newpage

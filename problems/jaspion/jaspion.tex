Em 1985 estréia na TV Japonesa a série Kyojiu Tokusou Jaspion (Investigador
Especial de Monstros Jaspion). A série chega ao Brasil alguns anos
depois com o título "O Fantástico Jaspion", e com ela nasce a fantasia de
polícia espacial em milhões de brasileirinhos. As crianças saíam da escola,
corriam pelas ruas (sem olhar se vinha carro), ligavam a TV e mergulhavam na
coragem, exemplo de pessoa, e incontestável sede por justiça do Fantástico
Jaspion. O comércio de gibis e as brigas por figurinhas no recreio da escola
estavam alcançando números históricos. Até então, tal sentimento só havia
sido estimulado com tanta intensidade pelo Chaves e a sua turma! Diante
dessa febre inter-galática, o inevitável aconteceu. Os produtores do Jaspion
ganharam o Nobel da Paz! Isso mesmo! Os produtores ganharam um Nobel. As
histórias do grandioso Jaspion estavam por todo canto. Agora as crianças
tinham um belíssimo exemplo para seguir. A paz mundial estava garantida! Não
precisávamos mais temer o monstrengo Satan Gos!

No Brasil havia uma criança que adorava as histórias do Jaspion! Antônio
Melhorança Capote Valente Junior carinhosamente apelidado de ACM, um menino da
zona sul de São Paulo que adorava cantar as músicas do grande herói. Ele era tão
fanático que chegou a comprar um dicionário de Japonês-Português e iniciou um
trabalho árduo de tradução. Entretanto, o trabalho ficou inacabado! Alguns
trechos da canção ainda precisam ser traduzidos. Neste momento você deve estar
se perguntando: qual é a minha tarefa neste fabuloso problema? Ok! Antes de
falar da sua tarefa, convide seu companheiro de equipe para mergulhar com você
no desfecho da história. Para isso, vamos falar mais um pouco sobre o nosso ACM.
Ele se formou em Ciência da Computação e hoje trabalha no mesmo escritório que
você. Pois é! Você trabalha como programador ao lado dessa figura! Como sabemos
que você gosta muito dele, temos certeza que vai aceitar a seguinte tarefa: dado
um dicionário Japonês-Português e uma letra de música, escreva um programa que
imprima a letra traduzida.


\subsection*{Entrada}

A primeira linha de um caso de testes contém um inteiro $T$ que indica o número de
instâncias subseqüentes. A primeira linha de cada instância contém dois inteiros
$M$ e $N$ ($1 \leq M \leq 1000000$, $1 \leq N \leq 1000$), que representam o número de palavras no
dicionário e o número de linhas na letra da música, respectivamente.

Os próximos $M$ pares de linhas contêm as traduções: a primeira linha de cada par
contém a palavra em Japonês, e a segunda linha contém a tradução para o
Português (que pode ter uma ou mais palavras). Todas as palavras usam apenas
letras minúsculas. Cada palavra em Japonês aparece apenas uma vez em cada
instância.

As próximas $N$ linhas contêm a letra da música. Cada linha da letra da música é
uma lista de palavras separadas por um espaço (todas as palavras consistem
apenas de letras minúsculas). Algumas podem estar vazias, mas nenhuma
linha possui espaços no início ou no final.

Nenhuma linha contém mais do que 80 letras.


\subsection*{Saída}

Para cada instância imprima as $N$ linhas traduzidas. As palavras que não estão no
dicionário devem ser impressas como aparecem na entrada. Imprima uma linha em
branco no final de cada tradução.

Nenhuma linha da saída contém mais do que 80 letras.

\begin{table}[!h]
\centering
\begin{tabular}{|l|l|}
\hline
\begin{minipage}[t]{3in}
\textbf{Exemplo de entrada}
\begin{verbatim}
1
4 3
galaxy
cara tossiu
kagayaku
canalha do
atsuki
alto que
yuushi
util
o galaxy
o galaxy
o kagayaku atsuki yuushi
\end{verbatim}
\vspace{1mm}
\end{minipage}
&

\begin{minipage}[t]{3in}
\textbf{Exemplo de saída}
\begin{verbatim}
o cara tossiu
o cara tossiu
o canalha do alto que util
\end{verbatim}
\vspace{1mm}
\end{minipage} \\
\hline
\end{tabular}
\end{table}

\newpage

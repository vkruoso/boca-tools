Og gosta muito de brincar com seus filhos. Seu jogo preferido é o jogo do maior,
de autoria própria. Este passatempo (no tempo das cavernas se tinha muito
tempo disponível para jogos) é jogado em dupla, Og e um dos seus
filhos. O jogo procede da seguinte forma: os dois participantes escolhem um
número de rodadas e, a cada rodada, cada participante diz um número de 0 até
10 em voz alta, sendo que o participante que falar o número mais alto ganha
um ponto (em caso de empate, ninguém ganha o ponto). No final das rodadas, os
pontos são contabilizados e o participante com o maior número de pontos
ganha.

Og e seus filhos gostam muito do jogo, mas se perdem na contagem dos pontos.
Você conseguirá ajudar Og a verificar a pontuação de uma lista de jogos?


\subsection*{Entrada}

A entrada é composta por vários casos de teste (partidas). Cada caso é iniciado
com um inteiro $N$ (de 0 até 10) representando o número de rodadas da partida,
sendo que o valor 0 representa o final da entrada e não deve ser processado.
Cada uma das próximas $N$ linhas contém dois inteiros, $A$ e $B$, onde $A$ é o
número escolhido pelo primeiro jogador e $B$ é o número escolhido pelo segundo
jogador ($0 \leq A, B \leq 10$).

\subsection*{Saída}

A saída deve ser composta por uma linha por caso de teste, contendo o número de
pontos de cada jogador, separados por um espaço.

\begin{table}[!h]
\centering
\begin{tabular}{|l|l|}
\hline
\begin{minipage}[t]{3in}
\textbf{Exemplo de entrada}
\begin{verbatim}
3
5 3
8 2
5 6
2
5 5
0 0
0
\end{verbatim}
\vspace{1mm}
\end{minipage}
&

\begin{minipage}[t]{3in}
\textbf{Exemplo de saída}
\begin{verbatim}
2 1
0 0
\end{verbatim}
\vspace{1mm}
\end{minipage} \\
\hline
\end{tabular}
\end{table}

\newpage

My birthday is coming up and traditionally I'm serving pie. Not just one pie,
   no, I have a number $N$ of them, of various tastes and of various sizes.
   $F$ of
   my friends are coming to my party and each of them gets a piece of pie. This
   should be one piece of one pie, not several small pieces since that looks
   messy. This piece can be one whole pie though.

   My friends are very annoying and if one of them gets a bigger piece than the
   others, they start complaining. Therefore all of them should get equally
   sized (but not necessarily equally shaped) pieces, even if this leads to
   some pie getting spoiled (which is better than spoiling the party). Of
   course, I want a piece of pie for myself too, and that piece should also be
   of the same size.

   What is the largest possible piece size all of us can get? All the pies are
   cylindrical in shape and they all have the same height $1$, but the radii of
   the pies can be different. 

\subsection*{Input}

One line with a positive integer: the number of test cases. Then for each test
case: 

\begin{enumerate}
    \item One line with two integers $N$ and $F$ with
    $1 \leq N, F \leq 10000$: the number of pies and the number of friends. 

    \item One line with $N$ integers $r_i$ with $1 \leq r_i \leq 10000$: the
    radii of the pies. 
\end{enumerate}

\subsection*{Output}

For each test case, output one line with the largest possible volume $V$ such
that me and my friends can all get a pie piece of size $V$. The answer should
be given as a floating point number with exactly six digits in the fractional
part.

\begin{table}[!h]
\centering
\begin{tabular}{|l|l|}
\hline
\begin{minipage}[t]{3in}
\textbf{Sample Input}
\begin{verbatim}
3
3 3
4 3 3
1 24
5
10 5
1 4 2 3 4 5 6 5 4 2
\end{verbatim}
\vspace{1mm}
\end{minipage}
&

\begin{minipage}[t]{3in}
\textbf{Sample Output}
\begin{verbatim}
25.132741
3.141593
50.265482
\end{verbatim}
\vspace{1mm}
\end{minipage} \\
\hline
\end{tabular}
\end{table}

\newpage

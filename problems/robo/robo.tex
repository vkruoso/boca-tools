Um dos esportes favoritos na Robolândia é o Rali dos Robôs. Este rali é
praticado em uma arena retangular gigante de $N$ linhas por $M$ colunas de células
quadradas. Algumas das células estão vazias, algumas contêm figurinhas da Copa
(muito apreciadas pelas inteligências artificiais da Robolândia) e algumas são
ocupadas por pilastras que sustentam o teto da arena. Em seu percurso os robôs
podem ocupar qualquer célula da arena, exceto as que contém pilastras, que
bloqueiam o seu movimento.

O percurso do robô na arena durante o rali é
determinado por uma sequência de instruções. Cada instrução é representada por
um dos seguintes caracteres: 'D', 'E' e 'F', significando, respectivamente,
``gire 90 graus para a direita'', ``gire 90 graus para a esquerda'' e ``ande uma
célula para a frente''. O robô começa o rali em uma posição inicial na arena e
segue fielmente a sequência de instruções dada (afinal, eles são robôs!).
Sempre que o robô ocupa uma célula que contém uma figurinha da Copa ele a
coleta. As figurinhas da Copa não são repostas, ou seja, cada figurinha pode
ser coletada uma unica vez. Quando um robô tenta andar para uma célula onde
existe uma pilastra ele patina, permanecendo na célula onde estava, com a
mesma orientação. O mesmo também acontece quando um robô tenta sair da
arena.

Dados o mapa da arena, descrevendo a posição de pilastras e figurinhas, e a
sequência de instruções de um robô, você deve escrever um programa para
determinar o número de figurinhas coletadas pelo robô.

\subsection*{Entrada}

A entrada contém vários casos de teste. A primeira linha de um caso de teste
contém três números inteiros $N$, $M$ e $S$
($1 \leq N, M \leq 100, 1 \leq S \leq 5 \times 10^4$),
separados por espaços em branco, indicando respectivamente o número de
linhas e o número de colunas da arena e o número de instruções para o robô.
Cada uma das $N$ linhas seguintes da entrada descreve uma linha de células da
arena e contém uma cadeia com $M$ caracteres. A primeira linha que aparece na
descrição da arena é a que está mais ao Norte; a primeira coluna que aparece
na descrição de uma linha de células da arena é a que está mais a Oeste.

Cada célula da arena pode conter um dos seguintes caracteres:
\begin{itemize}
\item '.' — célula normal;
\item '*' — célula que contém uma figurinha da Copa;
\item '\#' — célula que contém uma pilastra;
\item 'N', 'S', 'L', 'O' — célula onde o robô inicia o percurso (única na
      arena). A letra representa a orientação inicial do robô (Norte, Sul,
      Leste e Oeste, respectivamente).
\end{itemize}

A ultima linha da entrada contém uma sequência de $S$ caracteres dentre 'D', 'E'
e 'F', representando as instruções do robô.

O último caso de teste é seguido por uma linha que contém apenas três números
zero separados por um espaço em branco.

\subsection*{Saída}

Para cada rali descrito na entrada seu programa deve imprimir uma única linha
contendo um único inteiro, indicando o número de figurinhas que o robô
colecionou durante o rali.

\begin{table}[!h]
\centering
\begin{tabular}{|l|l|}
\hline
\begin{minipage}[t]{3in}
\textbf{Exemplo de entrada}
\begin{verbatim}
3 3 2
***
*N*
***
DE
4 4 5
...#
*#O.
*.*.
*.#.
FFEFF
10 10 20
....*.....
.......*..
.....*....
..*.#.....
...#N.*..*
...*......
..........
..........
..........
..........
FDFFFFFFEEFFFFFFEFDF
0 0 0
\end{verbatim}
\vspace{1mm}
\end{minipage}
&

\begin{minipage}[t]{3in}
\textbf{Exemplo de saída}
\begin{verbatim}
0
1
3
\end{verbatim}
\vspace{1mm}
\end{minipage} \\
\hline
\end{tabular}
\end{table}

\newpage

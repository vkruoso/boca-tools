A given triangle can be either equilateral (three sides of the same length),
scalene (three sides of different lengths), or isosceles (two sides of the
same length and a third side of a different length). It is a known fact
that points with all integer coordinates cannot be the vertices of an
equilateral triangle.

You are given a set of different points with integer coordinates on the XY
plane, such that no three points in the set lay on the same line. Your job
is to calculate how many of the possible choices of three points are the
vertices of an isosceles triangle.

\subsection*{Input}

There are several test cases. Each test case is given in several lines. The
first line of each test case contains an integer $N$ indicating the number
of points in the set $(3 \leq N  \leq 1000)$. Each of the next $N$ lines
describes a different point of the set using two integers $X$ and $Y$
separated by a single space $(1 \leq X, Y \leq 10^6)$; these values
represent the coordinates of the point on the XY plane. You may assume that
within each test case no two points have the same location and no three
points are collinear.

The last test case is followed by a line containing a single zero.

\subsection*{Output}

For each test case output a single line with a single integer indicating the
number of subsets of three points that are the vertices of an isosceles
triangle.

\begin{table}[!h]
\centering
\begin{tabular}{|l|l|}
\hline
\begin{minipage}[t]{3in}
\textbf{Sample Input}
\begin{verbatim}
5
1 2
2 1
2 2
1 1
1000 1000000
6
1000 1000
996 1003
996 997
1003 996
1003 1004
992 1000
0
\end{verbatim}
\vspace{1mm}
\end{minipage}
&

\begin{minipage}[t]{3in}
\textbf{Sample Output}
\begin{verbatim}
4
10
\end{verbatim}
\vspace{1mm}
\end{minipage} \\
\hline
\end{tabular}
\end{table}

\newpage

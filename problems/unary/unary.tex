Unary is a minimalistic Brainfuck dialect in which programs are written using only one token.

Brainfuck programs use 8 commands: "+", "-", "[", "]", "<", ">", "." and ","
(their meaning is not important for the purposes of this problem). Unary
programs are created from Brainfuck programs using the following algorithm.
First, replace each command with a corresponding binary code, using the
following conversion table:

\begin{table}[!h]
\centering
\begin{tabular}{ll}
\begin{minipage}[t]{3in}
\begin{itemize}
    \item '$<$' $\rightarrow$ 1000
    \item '$>$' $\rightarrow$ 1001
    \item '$+$' $\rightarrow$ 1010
    \item '$-$' $\rightarrow$ 1011
\end{itemize}
\vspace{1mm}
\end{minipage}
&
\begin{minipage}[t]{3in}
\begin{itemize}
   \item '$.$' $\rightarrow$ 1100
    \item '$,$' $\rightarrow$ 1101
    \item '$[$' $\rightarrow$ 1110
    \item '$]$' $\rightarrow$ 1111
\end{itemize}
\vspace{1mm}
\end{minipage} \\
\end{tabular}
\end{table}

Next, concatenate the resulting binary codes into one binary number in the same
order as in the program. Finally, write this number using unary numeral system
- this is the Unary program equivalent to the original Brainfuck one.

You are given a Brainfuck program. Your task is to calculate the size of the
equivalent Unary program, and print it modulo $1000003$ $(10^6 + 3)$.

\subsection*{Input}

The first line of the input contains the integer $T$ ($1 \leq T \leq 100$),
the number of test cases.
Each of the following T lines contains a test case, a program in brainfuck.
Each line will contain between 1 and 100 characters, inclusive. There are no
invalid characters.

\subsection*{Output}

Output, for each test case, the size of the equivalent Unary program modulo
$1000003$ $(10^6 + 3)$.

\begin{table}[!h]
\centering
\begin{tabular}{|l|l|}
\hline
\begin{minipage}[t]{3in}
\textbf{Sample Input}
\begin{verbatim}
2
,.
++++[>,.<-]
\end{verbatim}
\vspace{1mm}
\end{minipage}
&

\begin{minipage}[t]{3in}
\textbf{Sample Output}
\begin{verbatim}
220
61425
\end{verbatim}
\vspace{1mm}
\end{minipage} \\
\hline
\end{tabular}
\end{table}

\subsection*{Note}

To write a number $n$ in unary numeral system, one simply has to write 1 $n$
times. For example, 5 written in unary system will be 11111.

In the first example replacing Brainfuck commands with binary code will give us
1101 1100. After we concatenate the codes, we'll get 11011100 in binary system,
     or 220 in decimal. That's exactly the number of tokens in the equivalent
     Unary program.

\newpage

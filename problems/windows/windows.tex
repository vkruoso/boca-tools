Emma is not very tidy with the desktop of her computer. She has the habit of
opening windows on the screen and then not closing the application that created
them. The result, of course, is a very cluttered desktop with some windows just
peeking out from behind others and some completely hidden. Given that Emma
doesn't log off for days, this is a formidable mess. Your job is to determine
which window (if any) gets selected when Emma clicks on a certain position of
the screen.

Emma's screen has a resolution of $10^6$ by $10^6$. When each window opens its
position is given by the upper-left-hand corner, its width, and its height.
(Assume position (0,0) is the location of the pixel in the upper-left-hand
 corner of her desktop. So, the lower-right-hand pixel has location (999999,
 999999).)

\subsection*{Input}

Input for each test case is a sequence of desktop descriptions. Each description
consists of a line containing a positive integer $n$, the number of windows,
followed by $n$ lines, $n \leq 100$, describing windows in the order in which
Emma opened them, followed by a line containing a positive integer $m$,
the number of queries, followed by $m$ lines, each describing a query.
Each of the $n$ window description lines contains four integers $r$, $c$, $w$,
and $h$, where $(r, c)$ is the row and column of the upper left pixel of
the window, $0 \leq r, c \leq 999999$, and $w$ and $h$ are the width and
height of the window, in pixels, $1 \leq w, h$. All windows will lie
entirely on the desktop (i.e., no cropping). Each of the $m$ query
description lines contains two integers $cr$ and $cc$, the row and column
number of the location (which will be on the desktop). The last test
case is followed by a line containing $0$.

\subsection*{Output}

Using the format shown in the sample, for each test case, print the desktop
number, beginning with 1, followed by $m$ lines, one per query. The $i$-th line
should say either $window k$, where $k$ is the number of the window clicked on,
or $background$ if the query hit none of the windows. We assume that
windows are numbered consecutively in the order in which Emma opened
them, beginning with $1$. Note that querying a window does not bring that
window to the foreground on the screen.

\newpage

\begin{table}[!h]
\centering
\begin{tabular}{|l|l|}
\hline
\begin{minipage}[t]{3in}
\textbf{Sample Input}
\begin{verbatim}
3
1 2 3 3
2 3 2 2
3 4 2 2
4
3 5
1 2
4 2
3 3
2
5 10 2 10
100 100 100 100
2
5 13
100 101
0
\end{verbatim}
\vspace{1mm}
\end{minipage}
&

\begin{minipage}[t]{3in}
\textbf{Sample Output}
\begin{verbatim}
Desktop 1:
window 3
window 1
background
window 2
Desktop 2:
background
window 2
\end{verbatim}
\vspace{1mm}
\end{minipage} \\
\hline
\end{tabular}
\end{table}

\newpage

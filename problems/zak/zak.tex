Zak Galou é um famoso bruxo matador de monstros. Diz a lenda que existe uma
caverna escondida nos confins da selva contendo um tesouro milenar. Até hoje
nenhum aventureiro conseguiu recuperar o tesouro, pois ele é bem guardado por
terríveis monstros. Mas Zak Galou não é um aventureiro qualquer e decidiu
preparar-se para recuperar o tão sonhado tesouro.

Zak Galou dispõe de uma certa quantidade de mana (uma espécie de energia mágica)
e de uma lista de $M$ magias. Cada monstro tem um determinado número de pontos
de vida. Cada vez que Zak Galou lança uma magia contra um monstro, Zak gasta
uma certa quantidade de mana (o custo da magia) e inflinge um certo dano ao
monstro. O dano inflingido provoca a perda de pontos de vida do monstro (o
número de pontos perdidos depende da magia). Um monstro está morto
se tiver zero ou menos pontos de vida. Zak sempre luta contra um monstro a
cada vez. Como é um bruxo poderoso, ele pode usar a mesma magia várias
vezes, desde que possua a quantidade necessária de mana.

Em suas pesquisas, Zak Galou conseguiu o mapa do tesouro. A caverna é
representada como um conjunto de salões conectados por galerias. Os salões
são identificados sequencialmente de 1 a $N$. Zak sempre inicia no salão 1 e o
tesouro está sempre no salão $N$. Existem $K$ monstros identificados
seqüencialmente de 1 a $K$. Cada monstro vive em um salão, do qual não sai
(note que é possível que mais de um monstro viva no mesmo salão). Durante a
busca pelo tesouro, Zak Galou pode sair ou recuperar o tesouro de um salão
somente se o salão estiver vazio (sem monstro). Em outras palavras, Zak deve
sempre, antes de sair ou de recuperar o tesouro de um salão, matar o(s)
monstro(s) que lá viver(em).

Dadas as descrições das magias, dos monstros e da caverna, sua tarefa é
determinar a quantidade mínima inicial de mana necessária para que Zak Galou
consiga recuperar o tesouro.


\subsection*{Entrada}

A entrada contém vários casos de teste. A primeira linha de cada caso de teste
contém quatro inteiros $M, N, G$ e $K$, indicando respectivamente o número de magias
$(1 \leq M \leq 1000)$, de salões $(1 \leq N \leq 1000)$, de galerias $(0 \leq G
        \leq 10^6)$ e de monstros $(0 \leq K \leq 1000)$.

Cada uma das $M$ linhas seguintes descreve uma magia. A descrição de uma magia
contém dois números inteiros, a quantidade de mana consumida (entre 1 e 1000) e
o número de pontos de danos provocados (também entre 1 e 1000).

Em seguida, há $G$ linhas, cada uma descrevendo uma galeria. Uma galeria é
descrita por dois números inteiros $A$ e $B$ ($A \neq B$), representando os salões que a
galeria conecta. Zak pode utilizar a galeria nos dois sentidos, ou seja, para ir
de $A$ para $B$ ou de $B$ para $A$.

Finalmente, as últimas $K$ linhas de um caso de teste descrevem os monstros. A
descrição de um monstro contém dois números inteiros representando
respectivamente o salão no qual ele vive (entre 1 e $N$ inclusive) e o seu número
inicial de pontos de vida (entre 1 e 1000 inclusive).

O final da entrada é indicado por $M = N = G = K = 0$.

\subsection*{Saída}

Para cada caso de teste da entrada seu programa deve produzir uma linha na saída
contendo um número inteiro, a quantidade mínima inicial de mana necessária. Caso
não seja possível recuperar o tesouro, você deve imprimir $-1$.

\begin{table}[!h]
\centering
\begin{tabular}{|l|l|}
\hline
\begin{minipage}[t]{3in}
\textbf{Exemplo de Entrada}
\begin{verbatim}
3 4 4 2
7 10
13 20
25 50
1 2
2 4
1 3
3 4
2 125
3 160
3 4 4 1
7 10
13 20
25 50
1 2
2 4
1 3
3 4
2 125
1 3 1 1
1000 1000
1 2
3 1000
0 0 0 0
\end{verbatim}
\vspace{1mm}
\end{minipage}
&

\begin{minipage}[t]{3in}
\textbf{Exemplo de Saída}
\begin{verbatim}
70
0
-1
\end{verbatim}
\vspace{1mm}
\end{minipage} \\
\hline
\end{tabular}
\end{table}

\newpage
